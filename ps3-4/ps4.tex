% Options for packages loaded elsewhere
\PassOptionsToPackage{unicode}{hyperref}
\PassOptionsToPackage{hyphens}{url}
\PassOptionsToPackage{dvipsnames,svgnames,x11names}{xcolor}
%
\documentclass[
  letterpaper,
  abstract=true]{scrartcl}

\usepackage{amsmath,amssymb}
\usepackage{iftex}
\ifPDFTeX
  \usepackage[T1]{fontenc}
  \usepackage[utf8]{inputenc}
  \usepackage{textcomp} % provide euro and other symbols
\else % if luatex or xetex
  \usepackage{unicode-math}
  \defaultfontfeatures{Scale=MatchLowercase}
  \defaultfontfeatures[\rmfamily]{Ligatures=TeX,Scale=1}
\fi
\usepackage{lmodern}
\ifPDFTeX\else  
    % xetex/luatex font selection
\fi
% Use upquote if available, for straight quotes in verbatim environments
\IfFileExists{upquote.sty}{\usepackage{upquote}}{}
\IfFileExists{microtype.sty}{% use microtype if available
  \usepackage[]{microtype}
  \UseMicrotypeSet[protrusion]{basicmath} % disable protrusion for tt fonts
}{}
\makeatletter
\@ifundefined{KOMAClassName}{% if non-KOMA class
  \IfFileExists{parskip.sty}{%
    \usepackage{parskip}
  }{% else
    \setlength{\parindent}{0pt}
    \setlength{\parskip}{6pt plus 2pt minus 1pt}}
}{% if KOMA class
  \KOMAoptions{parskip=half}}
\makeatother
\usepackage{xcolor}
\usepackage[top=30mm,left=30mm,right=30mm,heightrounded]{geometry}
\setlength{\emergencystretch}{3em} % prevent overfull lines
\setcounter{secnumdepth}{-\maxdimen} % remove section numbering
% Make \paragraph and \subparagraph free-standing
\ifx\paragraph\undefined\else
  \let\oldparagraph\paragraph
  \renewcommand{\paragraph}[1]{\oldparagraph{#1}\mbox{}}
\fi
\ifx\subparagraph\undefined\else
  \let\oldsubparagraph\subparagraph
  \renewcommand{\subparagraph}[1]{\oldsubparagraph{#1}\mbox{}}
\fi


\providecommand{\tightlist}{%
  \setlength{\itemsep}{0pt}\setlength{\parskip}{0pt}}\usepackage{longtable,booktabs,array}
\usepackage{calc} % for calculating minipage widths
% Correct order of tables after \paragraph or \subparagraph
\usepackage{etoolbox}
\makeatletter
\patchcmd\longtable{\par}{\if@noskipsec\mbox{}\fi\par}{}{}
\makeatother
% Allow footnotes in longtable head/foot
\IfFileExists{footnotehyper.sty}{\usepackage{footnotehyper}}{\usepackage{footnote}}
\makesavenoteenv{longtable}
\usepackage{graphicx}
\makeatletter
\def\maxwidth{\ifdim\Gin@nat@width>\linewidth\linewidth\else\Gin@nat@width\fi}
\def\maxheight{\ifdim\Gin@nat@height>\textheight\textheight\else\Gin@nat@height\fi}
\makeatother
% Scale images if necessary, so that they will not overflow the page
% margins by default, and it is still possible to overwrite the defaults
% using explicit options in \includegraphics[width, height, ...]{}
\setkeys{Gin}{width=\maxwidth,height=\maxheight,keepaspectratio}
% Set default figure placement to htbp
\makeatletter
\def\fps@figure{htbp}
\makeatother

% TODO: Add custom LaTeX header directives here
\usepackage{float, booktabs, siunitx, orcidlink}
\setkomafont{disposition}{\bfseries}
\providecommand{\keywords}[1]
{
  \small	
  \textbf
{\textit{Keywords---}} #1
}
\makeatletter
\makeatother
\makeatletter
\makeatother
\makeatletter
\@ifpackageloaded{caption}{}{\usepackage{caption}}
\AtBeginDocument{%
\ifdefined\contentsname
  \renewcommand*\contentsname{Table of contents}
\else
  \newcommand\contentsname{Table of contents}
\fi
\ifdefined\listfigurename
  \renewcommand*\listfigurename{List of Figures}
\else
  \newcommand\listfigurename{List of Figures}
\fi
\ifdefined\listtablename
  \renewcommand*\listtablename{List of Tables}
\else
  \newcommand\listtablename{List of Tables}
\fi
\ifdefined\figurename
  \renewcommand*\figurename{Figure}
\else
  \newcommand\figurename{Figure}
\fi
\ifdefined\tablename
  \renewcommand*\tablename{Table}
\else
  \newcommand\tablename{Table}
\fi
}
\@ifpackageloaded{float}{}{\usepackage{float}}
\floatstyle{ruled}
\@ifundefined{c@chapter}{\newfloat{codelisting}{h}{lop}}{\newfloat{codelisting}{h}{lop}[chapter]}
\floatname{codelisting}{Listing}
\newcommand*\listoflistings{\listof{codelisting}{List of Listings}}
\makeatother
\makeatletter
\@ifpackageloaded{caption}{}{\usepackage{caption}}
\@ifpackageloaded{subcaption}{}{\usepackage{subcaption}}
\makeatother
\makeatletter
\@ifpackageloaded{tcolorbox}{}{\usepackage[skins,breakable]{tcolorbox}}
\makeatother
\makeatletter
\@ifundefined{shadecolor}{\definecolor{shadecolor}{rgb}{.97, .97, .97}}
\makeatother
\makeatletter
\makeatother
\makeatletter
\makeatother
\ifLuaTeX
  \usepackage{selnolig}  % disable illegal ligatures
\fi
\IfFileExists{bookmark.sty}{\usepackage{bookmark}}{\usepackage{hyperref}}
\IfFileExists{xurl.sty}{\usepackage{xurl}}{} % add URL line breaks if available
\urlstyle{same} % disable monospaced font for URLs
\hypersetup{
  pdftitle={The Democratic Critiques and Populism},
  colorlinks=true,
  linkcolor={blue},
  filecolor={Maroon},
  citecolor={Blue},
  urlcolor={Blue},
  pdfcreator={LaTeX via pandoc}}

\title{The Democratic Critiques and Populism}
\usepackage{etoolbox}
\makeatletter
\providecommand{\subtitle}[1]{% add subtitle to \maketitle
  \apptocmd{\@title}{\par {\large #1 \par}}{}{}
}
\makeatother
\subtitle{GV482 Problem Set - Game Theory}
\author{Dianyi Yang\textsuperscript{}}
\date{2024-02-02}

\begin{document}
\maketitle
\ifdefined\Shaded\renewenvironment{Shaded}{\begin{tcolorbox}[frame hidden, sharp corners, breakable, enhanced, interior hidden, boxrule=0pt, borderline west={3pt}{0pt}{shadecolor}]}{\end{tcolorbox}}\fi

\hypertarget{part-iii---the-populists-entry-decision-part-of-this-will-be-solved-on-thursday-2-february-the-rest-on-thursday-9-february}{%
\section{Part III - The Populist's entry decision (part of this will be
solved on Thursday 2 February, the rest on Thursday 9
February)}\label{part-iii---the-populists-entry-decision-part-of-this-will-be-solved-on-thursday-2-february-the-rest-on-thursday-9-february}}

In the previous part, we have computed the electoral decision of a voter
from group \(C\) as a function of her information. We use our previous
answers to compute \(P\)'s electoral chances and entry decision under
different scenarios.

\hypertarget{q7}{%
\subsection{Q7}\label{q7}}

In this question, we suppose that mainstream candidates find it optimal
to the commoners' preferred policy.

\begin{enumerate}
\def\labelenumi{(\alph{enumi})}
\item
  Using your answers to \textbf{Q3}, explain briefly why \(P\) never
  wins the election if he enters in this case.

  \color{blue}

  As we have shown in \textbf{Q3}, the probability of \(P\) winning the
  election is \(0\) if he enters.

  If \(P\) proposes the same platform (commoners' preferred policy) as
  \(A\) and \(B\), \(P\) will get no vote as both the commoners and
  elite will vote for the mainstream candidate with a positive valence
  shock. This corresponds to \textbf{Q3} (a).

  If \(P\) proposes a different platform, and

  \begin{itemize}
  \item
    the commoners and the elite's preferred policy is the same, \(P\)
    will get no vote as both the commoners and elite will vote for the
    mainstream candidate with a positive valence shock. This corresponds
    to \textbf{Q3} (b).
  \item
    the commoners and the elite's preferred policy is different, \(P\)
    will only get the elite's votes. This accounts for \(\sigma\) of the
    vote. Since this is less than 50\%, \(P\) will not win the election.
    This corresponds to \textbf{Q3} (c).
  \end{itemize}

  \color{black}
\item
  Explain why the populist never enters when
  \(\sigma+(1+\sigma)(1−p)<\frac{1}{2}\).

  \color{blue}

  As we have shown in \textbf{Q2} (e),
  \(\sigma+(1+\sigma)(1−p)<\frac{1}{2}\) implies that the mainstream
  candidates converge on the commoners preferred policy. As we discussed
  in Part (a), \(P\) has no chance of winning in this case. As there is
  a cost to entry, \(P\) will not enter:

  \[
  U_P(\text{enter})=-c<U_P(\text{not enter})=0
  \]

  \color{black}
\end{enumerate}

\hypertarget{q8}{%
\subsection{Q8}\label{q8}}

In question \protect\hyperlink{q8}{Q8} and \protect\hyperlink{q9}{Q9},
we turn to the case when mainstream candidates converge to the elite
citizens' preferred policy (\(x_A = x_B = \omega_E\) and this is
anticipated by commoners). In this question, we assume that
\(\alpha > 1/2\) and look at \(P\)'s entry decision then.

\begin{enumerate}
\def\labelenumi{(\alph{enumi})}
\item
  Assume that mainstream candidates propose \(x_A=x_B=0\) and the
  populist offers \(x_P =1\). Assume further than the state is
  \(\omega_C = 0\) (i.e., that is the optimal policy happens to be zero
  for the common voters). Using your answer to \textbf{Q5}(i), show that
  the vote share of candidate \(P\) is: \((1 − \sigma)(1 − p)\). Explain
  why P does not win the election then.

  \color{blue}

  The elite know their preference well. They will vote for one of the
  maintream candidates and not populist \(P\).

  As for the commoners,

  \begin{itemize}
  \item
    some receive signal of 1 (\(s_{i,C}=1\)). Their posterior belief
    (from \textbf{Q5} (j)) is:

    \[
    \mu(1,0,0,1)=\frac{(1-\alpha)p^2}{(1-\alpha)p^2+\alpha(1-p)^2}>\frac{1+|\delta|}{2}.
    \]

    They will vote for \(P\).
  \item
    Some receive signal of 0 (\(s_{i,C}=0\)). Their posterior belief is:

    \[
    \mu(0,0,0,1)=1-\alpha
    \]

    Since we assume \(\alpha>1/2\),
    \(\mu(0,0,0,1)=1-\alpha<\frac{1}{2}<\frac{1=|\delta|}{2}\). They
    will not vote for \(P\).
  \end{itemize}

  Therefore, only the commoners who receive signal of 1 will vote for
  \(P\). The probability of receiving this wrong signal
  \(Pr(s_{i,C}=1\mid \omega_C=0)=1-p\). They therefore account for
  \((1-\sigma)(1-p)\) of total voters. Since \(p>0.5\), this is smaller
  than 50\%. Other voters will unite around the mainstream candidate
  with a positive valence shock, who will receive more than 50\% of the
  vote. Therefore, \(P\) will not win the election.

  \color{black}
\item
  Assume that mainstream candidates propose \(x_A=x_B=0\) and the
  populist offers \(x_P =1\). Assume further than the state is
  \(\omega_C = 1\) (i.e., that is the optimal policy happens to be one
  for the common voters). Show that the vote share of candidate P is:
  \((1 − \sigma)p\). Explain why \(P\) loses the election then.

  \color{blue}

  The elite know their preference well. They will vote for one of the
  maintream candidates and not populist \(P\).

  As for the commoners,

  \begin{itemize}
  \item
    some receive signal of 1 (\(s_{i,C}=1\)). Their posterior belief
    (from \textbf{Q5} (j)) is:

    \[
    \mu(1,0,0,1)=\frac{(1-\alpha)p^2}{(1-\alpha)p^2+\alpha(1-p)^2}>\frac{1+|\delta|}{2}.
    \]

    They will vote for \(P\).
  \item
    Some receive signal of 0 (\(s_{i,C}=0\)). Their posterior belief is:

    \[
    \mu(0,0,0,1)=1-\alpha
    \]

    Since we assume \(\alpha>1/2\),
    \(\mu(0,0,0,1)=1-\alpha<\frac{1}{2}<\frac{1+|\delta|}{2}\). They
    will not vote for \(P\).
  \end{itemize}

  Therefore, only the commoners who receive signal of 1 will vote for
  \(P\). The probability of receiving this right signal
  \(Pr(s_{i,C}=1\mid \omega_C=1)=p\). They therefore account for
  \((1-\sigma)p\) of total voters. For the mainstream candidates to
  converge on the elite's preference, it requires:

  \begin{align*}
  \sigma+(1-\sigma)(1-p)&\geq\frac{1}{2} \\
  \sigma+(1-\sigma)-(1-\sigma)p&\geq\frac{1}{2} \\
  (1-\sigma)p&\leq\frac{1}{2}
  \end{align*}

  Therefore, \(P\) only receives less than half of the votes, whereas
  the mainstream candidate with a positive valence shock gathers more
  than half of the votes. \(P\) therefore loses.

  \color{black}
\item
  Assume that mainstream candidates propose \(x_A = x_B = 1\) and the
  populist offers \(x_P = 0\). Explain why P never wins the election.

  \color{blue}

  The elite know their preference well. They will vote for one of the
  maintream candidates and not populist \(P\).

  As for the commoners,

  \begin{itemize}
  \item
    some receive signal of 1 (\(s_{i,C}=1\)). Their posterior belief
    (from \textbf{Q6} (a)) is:

    \[
    \mu(1,1,1,0)=\alpha.
    \]

    As we have shown in \textbf{Q6} (d), commoners with signal 1 in this
    case only vote for \(P\) if and only if \(\alpha<1/2\). As we assume
    \(\alpha>1/2\) in this question, they will not vote for \(P\).
  \item
    Some receive signal of 0 (\(s_{i,C}=0\)). Their posterior belief is:

    \[
    \mu(0,1,1,0)=\frac{(1-p)^2\alpha}{(1-p)^2\alpha+p^2(1-\alpha)}
    \]

    As we have shown in \textbf{Q6} (d), commoners with signal 0 in this
    case always vote for \(P\).
  \end{itemize}

  Therefore, only commoners with signal 0 will vote for \(P\). The
  probability of receiving this signal is either \(p\) or \(1-p\). They
  therefore account for \((1-\sigma)p\) or \((1-\sigma)(1-p)\) of total
  voters. As we explored in Part (c), neither is greater than 50\% and
  one of the mainstream candidates gathers the majority of the vote.
  Therefore, \(P\) will not win the election.

  \color{black}
\item
  Explain briefly why P never enters.

  \color{blue}

  We have shown that, when \(\alpha>1/2\), the populist candidate \(P\)
  never wins the election. No matter which policy the mainstream
  candidates converge on and what policy the populist candidate \(P\)
  proposes, she has no chance of winning. Since there is a cost of
  running for election, \(P\) will never enter the election.

  \color{black}
\end{enumerate}

\hypertarget{q9}{%
\subsection{Q9}\label{q9}}

We now look at \(P\)'s electoral chances and entry decision when
\(\alpha<1/2\).

\begin{enumerate}
\def\labelenumi{(\alph{enumi})}
\item
  Assume that mainstream candidates propose \(x_A=x_B=0\) and the
  populist offers \(x_P=1\). Using your answer to \textbf{Q5}(i), show
  that the vote share of candidate \(P\) is: (\(1 − \sigma\)). Explain
  why \(P\) wins the election then.

  \color{blue}

  The elite will also vote for the mainstream candidates as their
  platform aligns with the elite's preference.

  As for the commoners,

  \begin{itemize}
  \item
    some receive signal of 1 (\(s_{i,C}=1\)). Their posterior belief
    (from \textbf{Q5} (j)) is:

    \[
    \mu(1,0,0,1)=\frac{(1-\alpha)p^2}{(1-\alpha)p^2+\alpha(1-p)^2}>\frac{1+|\delta|}{2}.
    \]

    They will vote for \(P\).
  \item
    Some receive signal of 0 (\(s_{i,C}=0\)). Their posterior belief is:

    \[
    \mu(0,0,0,1)=1-\alpha
    \]

    Since we assume \(\alpha<1/2\),
    \(\mu(0,0,0,1)=1-\alpha>\frac{1+|\delta|}{2}\). This is shown in
    \textbf{Q5} (j). They will also vote for \(P\).
  \end{itemize}

  Therefore, the \(P\) will enjoy all the votes, and only the votes of
  the commoners, who account for (\(1-\sigma\)) of all voters. Since
  there are more commoners than elite \(1-\sigma>1/2\), \(P\) will win
  the election.

  \color{black}
\item
  Assume that mainstream candidates propose \(x_A=x_B=1\) and the
  populist offers \(x_P=0\). Explain why \(P\) always wins the election
  then.

  \color{blue}

  The elite will also vote for the mainstream candidates as their
  platform aligns with the elite's preference.

  As for the commoners,

  \begin{itemize}
  \item
    some receive signal of 1 (\(s_{i,C}=1\)). Their posterior belief
    (from \textbf{Q6} (a)) is:

    \[
    \mu(1,1,1,0)=\alpha.
    \]

    As we have shown in \textbf{Q6} (d),
    \(\mu(1,1,1,0)=\alpha<\frac{1-|\delta|}{2}\) when \(\alpha<1/2\) and
    these commoners with signal of 1 will vote for \(P\).
  \item
    some receive signal of 0 (\(s_{i,C}=0\)). Their posterior belief is:

    \[
    \mu(0,1,1,0)=\frac{(1-p)^2\alpha}{(1-p)^2\alpha+p^2(1-\alpha)}
    \]

    We have shown in \textbf{Q6} (d) it is true that
    \(\mu(0,1,1,0)<\frac{1-|\delta|}{2}\) for all values of \(\alpha\).
    Therefore these commoners with signal of 1 will also vote for \(P\).
  \end{itemize}

  Therefore, the \(P\) will enjoy all the votes, and only the votes of
  the commoners, who account for (\(1-\sigma\)) of all voters. Since
  there are more commoners than elite \(1-\sigma>1/2\), \(P\) will win
  the election.
\end{enumerate}

We now need to compute \(P\)'s expected payoff if \(P\) enters. We
consider the populist's entry decision so we no longer assume that
\(x_P = s_{P,C}\) (this has to be a choice of the populist). However, we
still assume below that citizens anticipate that \(x_P = s_{P,C}\) if
the populist enters (we will see that this anticipation is correct
below). In turn, both the citizens and the populist anticipate that the
mainstream candidates propose \(x_A=x_B=\omega_E\) (again, we will see
that this anticipation is correct in {[}Part IV{]}).

To calculate \(P\)'s expected utility when he enters, we make use of
\(\rho(s_{P,C}, x_A, x_B)\) the populist's posterior that
\(\omega_C = 1\) given his signal and the mainstream candidates'
platform choices.

\begin{enumerate}
\def\labelenumi{(\alph{enumi})}
\setcounter{enumi}{2}
\tightlist
\item
  Show that the expected payoff of the populist candidate if he enters
  the race is:

  \begin{itemize}
  \item
    \(0 − c\) if \(x_P = x_A = x_B\)

    \color{blue}

    \color{black}
  \item
    \(\rho(s_{P,C}, 0, 0) − c\) if \(x_P = 1\) and \(x_A = x_B = 0\)
  \item
    \(1 − \rho(s_{P,C}, 1, 1) − c\) if \(x_P = 0\) and \(x_A = x_B = 1\)
  \end{itemize}
\end{enumerate}

\hypertarget{part-iv---the-mainstream-parties-adaptation}{%
\section{Part IV - The mainstream parties'
adaptation}\label{part-iv---the-mainstream-parties-adaptation}}



\end{document}
